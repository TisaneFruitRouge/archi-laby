\documentclass{report}
\usepackage{listings}
\usepackage{mips}
\usepackage{color}


\definecolor{CommentGreen}{rgb}{0,.6,0}
\lstset{
	language=[mips]Assembler,
	backgroundcolor=\color[RGB]{245,245,244},
	escapechar=@, % include LaTeX code between `@' characters
	keepspaces=true,   % needed to preserve spacing with lstinline
	basicstyle=\footnotesize\ttfamily\bfseries,
	commentstyle=\color{CommentGreen},
	stringstyle=\color{cyan},
	showstringspaces=false,
	keywordstyle=[1]\color{blue},    % instructions
	keywordstyle=[2]\color{magenta}, % directives
	keywordstyle=[3]\color{red},     % registers
}


\author{Vincent WENDLING}
\title{Projet Labyrithe}

\begin{document}

\chapter{Introducion}
Pour commencer, moi et mon binôme Germain nous sommes mis d'accord sur le partage des tâches. Il s'occuperait de la génération du labyrinthe, quand à moi, je me chargerais de la résolution. 


\chapter{Prendre en entrée la labyrithe}
La première question à se poser est la suivante: \textbf{Comment charger le fichier texte du labyrinthe dans le programme ?}
\newline
\newline
Ce processus ce passse en deux partie : 

\begin{enumerate}
	\item Le chargement du fichier texte
	\item La lecture du fichier texte
\end{enumerate}


Ces deux partie feront intervenir des syscall.

\paragraph{Chargement du fichier} 
Premièrement, nous voulons charger le fichier, cela pourra se faire à l'aide du code suivant :
\begin{lstlisting}[language={[mips]Assembler}]
	syscall $v0 13
	li $a1 3
\end{lstlisting}

\end{document}