\documentclass{report}
\usepackage{listings}
\usepackage{mips}
\usepackage{color}


\definecolor{CommentGreen}{rgb}{0,.6,0}
\lstset{
	language=[mips]Assembler,
	backgroundcolor=\color[RGB]{245,245,244},
	escapechar=@, % include LaTeX code between `@' characters
	keepspaces=true,   % needed to preserve spacing with lstinline
	basicstyle=\footnotesize\ttfamily\bfseries,
	commentstyle=\color{CommentGreen},
	stringstyle=\color{cyan},
	showstringspaces=false,
	keywordstyle=[1]\color{blue},    % instructions
	keywordstyle=[2]\color{magenta}, % directives
	keywordstyle=[3]\color{red},     % registers
}


\author{Vincent WENDLING}
\title{Projet Labyrithe}

\begin{document}

\chapter{Introducion}
Pour commencer, moi et mon binôme Germain nous sommes mis d'accord sur le partage des tâches. Il s'occuperait de la génération du labyrinthe, quand à moi, je me chargerais de la résolution. 


\chapter{Charger et lire le fichier texte}
La première question à se poser est la suivante: \textbf{Comment lire le fichier texte du labyrinthe dans le programme ?}
\newline
\newline
Ce processus ce passse en deux partie : 

\begin{enumerate}
	\item Le chargement du fichier texte
	\item La lecture du fichier texte
\end{enumerate}


Ces deux partie feront intervenir des syscall.

\paragraph{Chargement du fichier} 
Premièrement, nous voulons charger le fichier, cela pourra se faire à l'aide du syscall 13, selon le code suivant :
\begin{lstlisting}[language={[mips]Assembler}]
	li   $v0, 13       		  # syscall pour ouvrir un fichier
	la   $a0, nomFichier      # nom du fichier
	li   $a1, 0               # mode d'ouverture (lecture, ecriture etc..)
	li   $a2, 0               # argument ignoré par Mars il semblerait
	syscall                   
	move $s0, $v0             # sauvegarde du "file descriptor" dans $s0 
\end{lstlisting}

Expliquons cette partie de code :
\begin{itemize}
	\item La première ligne place juste dans le register \textbf{\$v0} le nombre 13, qui nous servira donc à faire le syscall approprié.
	\item Viennent ensuite les arguments : le premier, dans le registre \textbf{\$a0}, est le nom du ficher que nous souhaitons charger.
	\item Le deuxième, dans le registre \textbf{\$a1}, est le \textit{flag}. Le \textit{flag} correspond au ``mode'' dans lequelle sera ouvert le fichier :
		\begin{itemize}
			\item 0 = lecture seule
			\item 1 = écriture seule avec création
			\item 9 = écriture seule avec création et ajout
		\end{itemize}  
	\item Enfin, le dernier argument, dans le registre \textbf{\$a0}, correspond au \textit{mode}. Cependant, il semble que cet argument ne soit pas pris en compte par le compilateur Mars.
	\item Ensuite vient le syscall
	\item Enfin, nous sauvegardons le resultats de ce syscall dans le registre \textbf{\$a0}. Ce resultat est un \textit{file desciptor} qui nous servira à lire le fichier. 
\end{itemize}

Une fois avoir charger le fichier, nous exécuterons le code suivant afin de pouvoir lire le fichier et utiliser son contenu :
\begin{lstlisting}[language={[mips]Assembler}]
	li   $v0, 14              # syscall pour lire un fichier
	move $a0, $s0             # file descirpor
	la   $a1, buffer          # adress du buffer duquel lire
	li   $a2,  1000           # taille du buffer (hardcoded)
	syscall 
\end{lstlisting}
Expliquons ce code :
\begin{itemize}
	\item Premierement, nous passons la valeur 14 dans le registre \textbf{\$v0} afin de faire le syscall approprié.
	\item Ensuite, nous passons le \textit{file descriptor} reçu précedement un argument, dans \textbf{\$a0}
	\item Nous passons ensuite dans le registre \textbf{\$a1} l'adresse du \textit{buffer}, la zone mémoire depuis laquelle nous pourrons lire le fichier
	\item Enfin, nous passons le taille de ce \textit{buffer} en argument au registre \textbf{\$a2}. Celle-ci a été définie arbitrairement.
\end{itemize}

A présent, nous pouvons donc lire le fichier texte depuis la variable \textbf{buffer}. Cependant, afin de pouvoir parcourir et utiliser les valeurs données par le fichier texte, nous allons stocker ces valeurs dans un tableau

\chapter{Stocker les valeurs dans un tableau et parcourir ce tableau}


\end{document}



